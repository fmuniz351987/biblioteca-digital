
% Default to the notebook output style

    


% Inherit from the specified cell style.




    
\documentclass[11pt]{article}

    
    
    \usepackage[T1]{fontenc}
    % Nicer default font (+ math font) than Computer Modern for most use cases
    \usepackage{mathpazo}

    % Basic figure setup, for now with no caption control since it's done
    % automatically by Pandoc (which extracts ![](path) syntax from Markdown).
    \usepackage{graphicx}
    % We will generate all images so they have a width \maxwidth. This means
    % that they will get their normal width if they fit onto the page, but
    % are scaled down if they would overflow the margins.
    \makeatletter
    \def\maxwidth{\ifdim\Gin@nat@width>\linewidth\linewidth
    \else\Gin@nat@width\fi}
    \makeatother
    \let\Oldincludegraphics\includegraphics
    % Set max figure width to be 80% of text width, for now hardcoded.
    \renewcommand{\includegraphics}[1]{\Oldincludegraphics[width=.8\maxwidth]{#1}}
    % Ensure that by default, figures have no caption (until we provide a
    % proper Figure object with a Caption API and a way to capture that
    % in the conversion process - todo).
    \usepackage{caption}
    \DeclareCaptionLabelFormat{nolabel}{}
    \captionsetup{labelformat=nolabel}

    \usepackage{adjustbox} % Used to constrain images to a maximum size 
    \usepackage{xcolor} % Allow colors to be defined
    \usepackage{enumerate} % Needed for markdown enumerations to work
    \usepackage{geometry} % Used to adjust the document margins
    \usepackage{amsmath} % Equations
    \usepackage{amssymb} % Equations
    \usepackage{textcomp} % defines textquotesingle
    % Hack from http://tex.stackexchange.com/a/47451/13684:
    \AtBeginDocument{%
        \def\PYZsq{\textquotesingle}% Upright quotes in Pygmentized code
    }
    \usepackage{upquote} % Upright quotes for verbatim code
    \usepackage{eurosym} % defines \euro
    \usepackage[mathletters]{ucs} % Extended unicode (utf-8) support
    \usepackage[utf8x]{inputenc} % Allow utf-8 characters in the tex document
    \usepackage{fancyvrb} % verbatim replacement that allows latex
    \usepackage{grffile} % extends the file name processing of package graphics 
                         % to support a larger range 
    % The hyperref package gives us a pdf with properly built
    % internal navigation ('pdf bookmarks' for the table of contents,
    % internal cross-reference links, web links for URLs, etc.)
    \usepackage{hyperref}
    \usepackage{longtable} % longtable support required by pandoc >1.10
    \usepackage{booktabs}  % table support for pandoc > 1.12.2
    \usepackage[inline]{enumitem} % IRkernel/repr support (it uses the enumerate* environment)
    \usepackage[normalem]{ulem} % ulem is needed to support strikethroughs (\sout)
                                % normalem makes italics be italics, not underlines
    \usepackage{mathrsfs}
    

    
    
    % Colors for the hyperref package
    \definecolor{urlcolor}{rgb}{0,.145,.698}
    \definecolor{linkcolor}{rgb}{.71,0.21,0.01}
    \definecolor{citecolor}{rgb}{.12,.54,.11}

    % ANSI colors
    \definecolor{ansi-black}{HTML}{3E424D}
    \definecolor{ansi-black-intense}{HTML}{282C36}
    \definecolor{ansi-red}{HTML}{E75C58}
    \definecolor{ansi-red-intense}{HTML}{B22B31}
    \definecolor{ansi-green}{HTML}{00A250}
    \definecolor{ansi-green-intense}{HTML}{007427}
    \definecolor{ansi-yellow}{HTML}{DDB62B}
    \definecolor{ansi-yellow-intense}{HTML}{B27D12}
    \definecolor{ansi-blue}{HTML}{208FFB}
    \definecolor{ansi-blue-intense}{HTML}{0065CA}
    \definecolor{ansi-magenta}{HTML}{D160C4}
    \definecolor{ansi-magenta-intense}{HTML}{A03196}
    \definecolor{ansi-cyan}{HTML}{60C6C8}
    \definecolor{ansi-cyan-intense}{HTML}{258F8F}
    \definecolor{ansi-white}{HTML}{C5C1B4}
    \definecolor{ansi-white-intense}{HTML}{A1A6B2}
    \definecolor{ansi-default-inverse-fg}{HTML}{FFFFFF}
    \definecolor{ansi-default-inverse-bg}{HTML}{000000}

    % commands and environments needed by pandoc snippets
    % extracted from the output of `pandoc -s`
    \providecommand{\tightlist}{%
      \setlength{\itemsep}{0pt}\setlength{\parskip}{0pt}}
    \DefineVerbatimEnvironment{Highlighting}{Verbatim}{commandchars=\\\{\}}
    % Add ',fontsize=\small' for more characters per line
    \newenvironment{Shaded}{}{}
    \newcommand{\KeywordTok}[1]{\textcolor[rgb]{0.00,0.44,0.13}{\textbf{{#1}}}}
    \newcommand{\DataTypeTok}[1]{\textcolor[rgb]{0.56,0.13,0.00}{{#1}}}
    \newcommand{\DecValTok}[1]{\textcolor[rgb]{0.25,0.63,0.44}{{#1}}}
    \newcommand{\BaseNTok}[1]{\textcolor[rgb]{0.25,0.63,0.44}{{#1}}}
    \newcommand{\FloatTok}[1]{\textcolor[rgb]{0.25,0.63,0.44}{{#1}}}
    \newcommand{\CharTok}[1]{\textcolor[rgb]{0.25,0.44,0.63}{{#1}}}
    \newcommand{\StringTok}[1]{\textcolor[rgb]{0.25,0.44,0.63}{{#1}}}
    \newcommand{\CommentTok}[1]{\textcolor[rgb]{0.38,0.63,0.69}{\textit{{#1}}}}
    \newcommand{\OtherTok}[1]{\textcolor[rgb]{0.00,0.44,0.13}{{#1}}}
    \newcommand{\AlertTok}[1]{\textcolor[rgb]{1.00,0.00,0.00}{\textbf{{#1}}}}
    \newcommand{\FunctionTok}[1]{\textcolor[rgb]{0.02,0.16,0.49}{{#1}}}
    \newcommand{\RegionMarkerTok}[1]{{#1}}
    \newcommand{\ErrorTok}[1]{\textcolor[rgb]{1.00,0.00,0.00}{\textbf{{#1}}}}
    \newcommand{\NormalTok}[1]{{#1}}
    
    % Additional commands for more recent versions of Pandoc
    \newcommand{\ConstantTok}[1]{\textcolor[rgb]{0.53,0.00,0.00}{{#1}}}
    \newcommand{\SpecialCharTok}[1]{\textcolor[rgb]{0.25,0.44,0.63}{{#1}}}
    \newcommand{\VerbatimStringTok}[1]{\textcolor[rgb]{0.25,0.44,0.63}{{#1}}}
    \newcommand{\SpecialStringTok}[1]{\textcolor[rgb]{0.73,0.40,0.53}{{#1}}}
    \newcommand{\ImportTok}[1]{{#1}}
    \newcommand{\DocumentationTok}[1]{\textcolor[rgb]{0.73,0.13,0.13}{\textit{{#1}}}}
    \newcommand{\AnnotationTok}[1]{\textcolor[rgb]{0.38,0.63,0.69}{\textbf{\textit{{#1}}}}}
    \newcommand{\CommentVarTok}[1]{\textcolor[rgb]{0.38,0.63,0.69}{\textbf{\textit{{#1}}}}}
    \newcommand{\VariableTok}[1]{\textcolor[rgb]{0.10,0.09,0.49}{{#1}}}
    \newcommand{\ControlFlowTok}[1]{\textcolor[rgb]{0.00,0.44,0.13}{\textbf{{#1}}}}
    \newcommand{\OperatorTok}[1]{\textcolor[rgb]{0.40,0.40,0.40}{{#1}}}
    \newcommand{\BuiltInTok}[1]{{#1}}
    \newcommand{\ExtensionTok}[1]{{#1}}
    \newcommand{\PreprocessorTok}[1]{\textcolor[rgb]{0.74,0.48,0.00}{{#1}}}
    \newcommand{\AttributeTok}[1]{\textcolor[rgb]{0.49,0.56,0.16}{{#1}}}
    \newcommand{\InformationTok}[1]{\textcolor[rgb]{0.38,0.63,0.69}{\textbf{\textit{{#1}}}}}
    \newcommand{\WarningTok}[1]{\textcolor[rgb]{0.38,0.63,0.69}{\textbf{\textit{{#1}}}}}
    
    
    % Define a nice break command that doesn't care if a line doesn't already
    % exist.
    \def\br{\hspace*{\fill} \\* }
    % Math Jax compatibility definitions
    \def\gt{>}
    \def\lt{<}
    \let\Oldtex\TeX
    \let\Oldlatex\LaTeX
    \renewcommand{\TeX}{\textrm{\Oldtex}}
    \renewcommand{\LaTeX}{\textrm{\Oldlatex}}
    % Document parameters
    % Document title
    \title{Biblioteca Digital}
    \author{Felipe Ribas Muniz - Matrícula 2018096227}
    
    
    
    
    

    % Pygments definitions
    
\makeatletter
\def\PY@reset{\let\PY@it=\relax \let\PY@bf=\relax%
    \let\PY@ul=\relax \let\PY@tc=\relax%
    \let\PY@bc=\relax \let\PY@ff=\relax}
\def\PY@tok#1{\csname PY@tok@#1\endcsname}
\def\PY@toks#1+{\ifx\relax#1\empty\else%
    \PY@tok{#1}\expandafter\PY@toks\fi}
\def\PY@do#1{\PY@bc{\PY@tc{\PY@ul{%
    \PY@it{\PY@bf{\PY@ff{#1}}}}}}}
\def\PY#1#2{\PY@reset\PY@toks#1+\relax+\PY@do{#2}}

\expandafter\def\csname PY@tok@w\endcsname{\def\PY@tc##1{\textcolor[rgb]{0.73,0.73,0.73}{##1}}}
\expandafter\def\csname PY@tok@c\endcsname{\let\PY@it=\textit\def\PY@tc##1{\textcolor[rgb]{0.25,0.50,0.50}{##1}}}
\expandafter\def\csname PY@tok@cp\endcsname{\def\PY@tc##1{\textcolor[rgb]{0.74,0.48,0.00}{##1}}}
\expandafter\def\csname PY@tok@k\endcsname{\let\PY@bf=\textbf\def\PY@tc##1{\textcolor[rgb]{0.00,0.50,0.00}{##1}}}
\expandafter\def\csname PY@tok@kp\endcsname{\def\PY@tc##1{\textcolor[rgb]{0.00,0.50,0.00}{##1}}}
\expandafter\def\csname PY@tok@kt\endcsname{\def\PY@tc##1{\textcolor[rgb]{0.69,0.00,0.25}{##1}}}
\expandafter\def\csname PY@tok@o\endcsname{\def\PY@tc##1{\textcolor[rgb]{0.40,0.40,0.40}{##1}}}
\expandafter\def\csname PY@tok@ow\endcsname{\let\PY@bf=\textbf\def\PY@tc##1{\textcolor[rgb]{0.67,0.13,1.00}{##1}}}
\expandafter\def\csname PY@tok@nb\endcsname{\def\PY@tc##1{\textcolor[rgb]{0.00,0.50,0.00}{##1}}}
\expandafter\def\csname PY@tok@nf\endcsname{\def\PY@tc##1{\textcolor[rgb]{0.00,0.00,1.00}{##1}}}
\expandafter\def\csname PY@tok@nc\endcsname{\let\PY@bf=\textbf\def\PY@tc##1{\textcolor[rgb]{0.00,0.00,1.00}{##1}}}
\expandafter\def\csname PY@tok@nn\endcsname{\let\PY@bf=\textbf\def\PY@tc##1{\textcolor[rgb]{0.00,0.00,1.00}{##1}}}
\expandafter\def\csname PY@tok@ne\endcsname{\let\PY@bf=\textbf\def\PY@tc##1{\textcolor[rgb]{0.82,0.25,0.23}{##1}}}
\expandafter\def\csname PY@tok@nv\endcsname{\def\PY@tc##1{\textcolor[rgb]{0.10,0.09,0.49}{##1}}}
\expandafter\def\csname PY@tok@no\endcsname{\def\PY@tc##1{\textcolor[rgb]{0.53,0.00,0.00}{##1}}}
\expandafter\def\csname PY@tok@nl\endcsname{\def\PY@tc##1{\textcolor[rgb]{0.63,0.63,0.00}{##1}}}
\expandafter\def\csname PY@tok@ni\endcsname{\let\PY@bf=\textbf\def\PY@tc##1{\textcolor[rgb]{0.60,0.60,0.60}{##1}}}
\expandafter\def\csname PY@tok@na\endcsname{\def\PY@tc##1{\textcolor[rgb]{0.49,0.56,0.16}{##1}}}
\expandafter\def\csname PY@tok@nt\endcsname{\let\PY@bf=\textbf\def\PY@tc##1{\textcolor[rgb]{0.00,0.50,0.00}{##1}}}
\expandafter\def\csname PY@tok@nd\endcsname{\def\PY@tc##1{\textcolor[rgb]{0.67,0.13,1.00}{##1}}}
\expandafter\def\csname PY@tok@s\endcsname{\def\PY@tc##1{\textcolor[rgb]{0.73,0.13,0.13}{##1}}}
\expandafter\def\csname PY@tok@sd\endcsname{\let\PY@it=\textit\def\PY@tc##1{\textcolor[rgb]{0.73,0.13,0.13}{##1}}}
\expandafter\def\csname PY@tok@si\endcsname{\let\PY@bf=\textbf\def\PY@tc##1{\textcolor[rgb]{0.73,0.40,0.53}{##1}}}
\expandafter\def\csname PY@tok@se\endcsname{\let\PY@bf=\textbf\def\PY@tc##1{\textcolor[rgb]{0.73,0.40,0.13}{##1}}}
\expandafter\def\csname PY@tok@sr\endcsname{\def\PY@tc##1{\textcolor[rgb]{0.73,0.40,0.53}{##1}}}
\expandafter\def\csname PY@tok@ss\endcsname{\def\PY@tc##1{\textcolor[rgb]{0.10,0.09,0.49}{##1}}}
\expandafter\def\csname PY@tok@sx\endcsname{\def\PY@tc##1{\textcolor[rgb]{0.00,0.50,0.00}{##1}}}
\expandafter\def\csname PY@tok@m\endcsname{\def\PY@tc##1{\textcolor[rgb]{0.40,0.40,0.40}{##1}}}
\expandafter\def\csname PY@tok@gh\endcsname{\let\PY@bf=\textbf\def\PY@tc##1{\textcolor[rgb]{0.00,0.00,0.50}{##1}}}
\expandafter\def\csname PY@tok@gu\endcsname{\let\PY@bf=\textbf\def\PY@tc##1{\textcolor[rgb]{0.50,0.00,0.50}{##1}}}
\expandafter\def\csname PY@tok@gd\endcsname{\def\PY@tc##1{\textcolor[rgb]{0.63,0.00,0.00}{##1}}}
\expandafter\def\csname PY@tok@gi\endcsname{\def\PY@tc##1{\textcolor[rgb]{0.00,0.63,0.00}{##1}}}
\expandafter\def\csname PY@tok@gr\endcsname{\def\PY@tc##1{\textcolor[rgb]{1.00,0.00,0.00}{##1}}}
\expandafter\def\csname PY@tok@ge\endcsname{\let\PY@it=\textit}
\expandafter\def\csname PY@tok@gs\endcsname{\let\PY@bf=\textbf}
\expandafter\def\csname PY@tok@gp\endcsname{\let\PY@bf=\textbf\def\PY@tc##1{\textcolor[rgb]{0.00,0.00,0.50}{##1}}}
\expandafter\def\csname PY@tok@go\endcsname{\def\PY@tc##1{\textcolor[rgb]{0.53,0.53,0.53}{##1}}}
\expandafter\def\csname PY@tok@gt\endcsname{\def\PY@tc##1{\textcolor[rgb]{0.00,0.27,0.87}{##1}}}
\expandafter\def\csname PY@tok@err\endcsname{\def\PY@bc##1{\setlength{\fboxsep}{0pt}\fcolorbox[rgb]{1.00,0.00,0.00}{1,1,1}{\strut ##1}}}
\expandafter\def\csname PY@tok@kc\endcsname{\let\PY@bf=\textbf\def\PY@tc##1{\textcolor[rgb]{0.00,0.50,0.00}{##1}}}
\expandafter\def\csname PY@tok@kd\endcsname{\let\PY@bf=\textbf\def\PY@tc##1{\textcolor[rgb]{0.00,0.50,0.00}{##1}}}
\expandafter\def\csname PY@tok@kn\endcsname{\let\PY@bf=\textbf\def\PY@tc##1{\textcolor[rgb]{0.00,0.50,0.00}{##1}}}
\expandafter\def\csname PY@tok@kr\endcsname{\let\PY@bf=\textbf\def\PY@tc##1{\textcolor[rgb]{0.00,0.50,0.00}{##1}}}
\expandafter\def\csname PY@tok@bp\endcsname{\def\PY@tc##1{\textcolor[rgb]{0.00,0.50,0.00}{##1}}}
\expandafter\def\csname PY@tok@fm\endcsname{\def\PY@tc##1{\textcolor[rgb]{0.00,0.00,1.00}{##1}}}
\expandafter\def\csname PY@tok@vc\endcsname{\def\PY@tc##1{\textcolor[rgb]{0.10,0.09,0.49}{##1}}}
\expandafter\def\csname PY@tok@vg\endcsname{\def\PY@tc##1{\textcolor[rgb]{0.10,0.09,0.49}{##1}}}
\expandafter\def\csname PY@tok@vi\endcsname{\def\PY@tc##1{\textcolor[rgb]{0.10,0.09,0.49}{##1}}}
\expandafter\def\csname PY@tok@vm\endcsname{\def\PY@tc##1{\textcolor[rgb]{0.10,0.09,0.49}{##1}}}
\expandafter\def\csname PY@tok@sa\endcsname{\def\PY@tc##1{\textcolor[rgb]{0.73,0.13,0.13}{##1}}}
\expandafter\def\csname PY@tok@sb\endcsname{\def\PY@tc##1{\textcolor[rgb]{0.73,0.13,0.13}{##1}}}
\expandafter\def\csname PY@tok@sc\endcsname{\def\PY@tc##1{\textcolor[rgb]{0.73,0.13,0.13}{##1}}}
\expandafter\def\csname PY@tok@dl\endcsname{\def\PY@tc##1{\textcolor[rgb]{0.73,0.13,0.13}{##1}}}
\expandafter\def\csname PY@tok@s2\endcsname{\def\PY@tc##1{\textcolor[rgb]{0.73,0.13,0.13}{##1}}}
\expandafter\def\csname PY@tok@sh\endcsname{\def\PY@tc##1{\textcolor[rgb]{0.73,0.13,0.13}{##1}}}
\expandafter\def\csname PY@tok@s1\endcsname{\def\PY@tc##1{\textcolor[rgb]{0.73,0.13,0.13}{##1}}}
\expandafter\def\csname PY@tok@mb\endcsname{\def\PY@tc##1{\textcolor[rgb]{0.40,0.40,0.40}{##1}}}
\expandafter\def\csname PY@tok@mf\endcsname{\def\PY@tc##1{\textcolor[rgb]{0.40,0.40,0.40}{##1}}}
\expandafter\def\csname PY@tok@mh\endcsname{\def\PY@tc##1{\textcolor[rgb]{0.40,0.40,0.40}{##1}}}
\expandafter\def\csname PY@tok@mi\endcsname{\def\PY@tc##1{\textcolor[rgb]{0.40,0.40,0.40}{##1}}}
\expandafter\def\csname PY@tok@il\endcsname{\def\PY@tc##1{\textcolor[rgb]{0.40,0.40,0.40}{##1}}}
\expandafter\def\csname PY@tok@mo\endcsname{\def\PY@tc##1{\textcolor[rgb]{0.40,0.40,0.40}{##1}}}
\expandafter\def\csname PY@tok@ch\endcsname{\let\PY@it=\textit\def\PY@tc##1{\textcolor[rgb]{0.25,0.50,0.50}{##1}}}
\expandafter\def\csname PY@tok@cm\endcsname{\let\PY@it=\textit\def\PY@tc##1{\textcolor[rgb]{0.25,0.50,0.50}{##1}}}
\expandafter\def\csname PY@tok@cpf\endcsname{\let\PY@it=\textit\def\PY@tc##1{\textcolor[rgb]{0.25,0.50,0.50}{##1}}}
\expandafter\def\csname PY@tok@c1\endcsname{\let\PY@it=\textit\def\PY@tc##1{\textcolor[rgb]{0.25,0.50,0.50}{##1}}}
\expandafter\def\csname PY@tok@cs\endcsname{\let\PY@it=\textit\def\PY@tc##1{\textcolor[rgb]{0.25,0.50,0.50}{##1}}}

\def\PYZbs{\char`\\}
\def\PYZus{\char`\_}
\def\PYZob{\char`\{}
\def\PYZcb{\char`\}}
\def\PYZca{\char`\^}
\def\PYZam{\char`\&}
\def\PYZlt{\char`\<}
\def\PYZgt{\char`\>}
\def\PYZsh{\char`\#}
\def\PYZpc{\char`\%}
\def\PYZdl{\char`\$}
\def\PYZhy{\char`\-}
\def\PYZsq{\char`\'}
\def\PYZdq{\char`\"}
\def\PYZti{\char`\~}
% for compatibility with earlier versions
\def\PYZat{@}
\def\PYZlb{[}
\def\PYZrb{]}
\makeatother


    % Exact colors from NB
    \definecolor{incolor}{rgb}{0.0, 0.0, 0.5}
    \definecolor{outcolor}{rgb}{0.545, 0.0, 0.0}



    
    % Prevent overflowing lines due to hard-to-break entities
    \sloppy 
    % Setup hyperref package
    \hypersetup{
      breaklinks=true,  % so long urls are correctly broken across lines
      colorlinks=true,
      urlcolor=urlcolor,
      linkcolor=linkcolor,
      citecolor=citecolor,
      }
    % Slightly bigger margins than the latex defaults
    
    \geometry{verbose,tmargin=1in,bmargin=1in,lmargin=1in,rmargin=1in}
    
    

    \begin{document}
    
    
    \maketitle
    
    

    
    \section{Introdução}\label{introduuxe7uxe3o}

O programa atual foi escrito em C++ para comparar diversas
implementações distintas do algoritmo de ordenação do quicksort. Nele,
foram analisados o quicksort clássico (QC), que usa como pivô o elemento
central de cada partição do vetor; o quicksort não recursivo usando como
pivô a mediana dos dois extremos e o elemento central (QNR); o quicksort
de mediana desses três elementos, mas implementado recursivamente (QM3);
o quicksort implementado usando o primeiro elemento da partição como
pivô (QPE); e quicksorts que, ao atingirem um percentual mínimo do vetor
sendo partido, passam a implementar inserção (1\%, 5\% e 10\%, ou QI1,
QI5 e QI10, respectivamente).

O quicksort usa a estratégia de dividir para conquistar, e, por
simplicidade de código, geralmente é implementado usando recursividade.
Ele quebra um vetor em duas partes e chama a si mesmo em cada parte (ou
partição) desse vetor, usando de forma arbitrária algum elemento dos
subvetores para repartir e chamar novamente de forma recursiva. Os
elementos maiores que o pivô são trocados com os menores quando estão na
partição errada. Assim, é esperado que os quicksorts que fazem uma
escolha mais cautelosa dos pivôs venham a lidar melhor com casos
extremos (como quando o vetor está ordenado de forma decrescente ou
crescente); por isso, é esperado que o QC, QM3 e o QNR apresentem bons
resultados, tanto nos casos de entrada aleatória quando no melhor e pior
caso. Os algoritmos de quicksort apresentam complexidade da ordem
O(n*log(n)), enquanto o de inserção possui complexidade maior, O(n²). É
esperado, portanto, que, quanto maior o percentual mínimo para iniciar a
inserção, mais demorado seja a execução. Os resultados são apresentados
ao final desse documento.

    \section{Implementação}\label{implementauxe7uxe3o}

O programa foi dividido nos seguintes módulos: - main: o programa
princial; - pilha: responsável por implementar a pilha usada pelo
algoritmo não-recursivo; - quicksort: uma classe responsável por
executar os algoritmos, além de manter a contagem de comparações e de
movimentações entre elementos do vetor, e de cronometrar o tempo de
execução; - vector\_generator: responsável por gerar vetores em ordem
crescente, decrescente ou aleatória; - argparser: responsável por fazer
o parsing dos argumentos de entrada.

    \section{Instruções de compilação e
execução}\label{instruuxe7uxf5es-de-compilauxe7uxe3o-e-execuuxe7uxe3o}

Para compilar o programa, basta estar no diretório raiz e digitar no
terminal: \texttt{bash\ make}

Para executar o programa, use:
\texttt{bash\ ./biblioteca.out\ \textless{}variacao\textgreater{}\ \textless{}tipo\textgreater{}\ \textless{}tamanho\textgreater{}\ {[}-p{]}}

Onde variacao é qual algoritmo se deseja usar (QC, QPE, QM3, QI1, QI5,
QI10 ou QNR), tipo é o tipo de vetor de entrada - aleatório, crescente
ou decrescente (Ale, OrdC, OrdD), tamanho é o tamanho do vetor de
entrada e {[}-p{]} é um parâmetro opcional que imprime os vetores usados
nos testes.

    \section{Análise experimental}\label{anuxe1lise-experimental}

Os diretórios aux e analise, localizados na pasta raiz, foram usados
para gerar a bateria de testes a ser executada e para plotar gráficos e
escrever o relatório final do projeto. Essa seção analisa como foi a
execução em cada caso, relativo ao tempo de duração, número de
movimentações e de comparações entre elementos do vetor, além de testar
nos casos de entrada aleatória, ordenada crescente e ordenada
decrescente, em cada algoritmo que foi implementado.

    \subsection{Comparação de tempo}\label{comparauxe7uxe3o-de-tempo}

    \begin{center}
    \adjustimage{max size={0.9\linewidth}{0.9\paperheight}}{graficos-comparativos_files/graficos-comparativos_10_0.png}
    \end{center}
    { \hspace*{\fill} \\}
    
    Com entrada aleatória, o quicksort não-recursivo apresenta o melhor
tempo de execução, sendo o único que não atingiu sequer um centésimo de
segundo para ordenar o vetor, mesmo com tamanho de 500k. O quicksort com
pivô no primeiro elemento (QPE), pivô de mediana de 3 elementos e o
quicksort clássico mostram comportamento similar, oscilando de 0,01 a
0,1 segundo de duração para ordenar o vetor de tamanho variando de 50k a
500k elementos. Os quicksorts mistos com inserção de 1\%, 5\% e 10\%
apresentam os piores tempos, com tempo aumentando proporcionalmente com
o aumento no percentual mínimo para iniciar o algoritmo de iserção.

    \begin{center}
    \adjustimage{max size={0.9\linewidth}{0.9\paperheight}}{graficos-comparativos_files/graficos-comparativos_12_0.png}
    \end{center}
    { \hspace*{\fill} \\}
    
    O quicksort com pivô no primeiro elemento apresenta, de longe, o pior
resultado. Para extrair a mediana dos tempos do QPE, foram necessárias
cerca de 8 horas de benchmarking. Também houve problema de estourar a
pilha apenas com esse algoritmo, criando-se a necessidade de aumentar o
tamanho da pilha para que ele pudesse executar apropriadamente. Ele foi
o único algoritmo a demorar um tempo na casa de 10\^{}2 segundos (alguns
chegaram a 400 segundos - mais de 10 minutos por vetor ordenado). O
quicksort não-recursivo, o quicksort de mediana 3 e o quicksort clássico
novamente classificam-se no mesmo grupo, mas dessa vez obtendo o segundo
melhor desempenho, ao invés do primeiro. O grupo dos algoritmos mistos
com inserção mostraram o melhor desempenho nesse caso.

    \begin{center}
    \adjustimage{max size={0.9\linewidth}{0.9\paperheight}}{graficos-comparativos_files/graficos-comparativos_14_0.png}
    \end{center}
    { \hspace*{\fill} \\}
    
    De forma similar ao caso do vetor ordenado crescente, o vetor ordenado
decrescente também resultou em três grupos: o quicksort com pivô no
primeiro elemento (QPE) mostrou o pior tempo; o grupo QNR/QM3/QC (grupo
2) mostrou o segundo melhor desempelho e os algoritmos mesclados com
inserção (grupo 3) mostraram o melhor desempenho por tempo. No entanto,
a melhoria no desemenho, tanto no caso da ordem crescente quanto da
ordem decrescente, não foi significativa quando comparam-se os grupos 2
com o 3; ela chega a ser apenas 10-15\% mais eficiente. Nesse caso,
quanto maior for o limite mínimo para iniciar o algoritmo de inserção,
mais rápida foi a execução.

    \subsection{Número de
movimentações}\label{nuxfamero-de-movimentauxe7uxf5es}

    \begin{center}
    \adjustimage{max size={0.9\linewidth}{0.9\paperheight}}{graficos-comparativos_files/graficos-comparativos_17_0.png}
    \end{center}
    { \hspace*{\fill} \\}
    
    Com vetores aleatórios, grupo de algoritmos 3 (os de inserção) mostraram
o maior número de movimentações de elementos dentro do vetor de entrada.
Quanto maior a taxa mínima para a inserção iniciar, maior foi o número
de movimentações. Os algoritmos clássico, QPE e mediana-3 mostraram uma
quantidade de movimentações relativamente pequena e estável. O
não-recursivo apresentou, de longe, o menor número de movimentações,
chegando a manter uma média de 1/10 de movimentações comparado ao grupo
de segundo melhor desempenho (!).

    \begin{center}
    \adjustimage{max size={0.9\linewidth}{0.9\paperheight}}{graficos-comparativos_files/graficos-comparativos_19_0.png}
    \end{center}
    { \hspace*{\fill} \\}
    
    O número de movimentações apresenta a menor variância entre um algoritmo
e outro quando o vetor está ordenado, de tal forma que as diferenças
aqui chegam a ser irrisórias, salva a exceção do QPE. Os algoritmos
mistos de inserção não mostram nenhuma diferença entre si no número de
movimentações, obtendo os melhores resultados, enquanto o QM3, QC e o
QNR apresentam uma quantidade ligeiramente maior de movimentações.

    \begin{center}
    \adjustimage{max size={0.9\linewidth}{0.9\paperheight}}{graficos-comparativos_files/graficos-comparativos_21_0.png}
    \end{center}
    { \hspace*{\fill} \\}
    
    Com o vetor ordenado de forma decrescente, o número de movimentações do
QPE equiparou-se ao dos mistos de inserção, sendo o grupo formado por
estes quatro algoritmos o de melhor desempenho. O QNR, QC e QM3
mostraram um desempenho ligeiramente menor, obtendo um número de
comparações um pouco maior, mas tendendo a equiparar-se ao outro grupo
na medida que o tamanho do vetor de entrada cresce.

    \subsection{Número de comparações}\label{nuxfamero-de-comparauxe7uxf5es}

    \begin{center}
    \adjustimage{max size={0.9\linewidth}{0.9\paperheight}}{graficos-comparativos_files/graficos-comparativos_24_0.png}
    \end{center}
    { \hspace*{\fill} \\}
    
    Com a entrada aleatória, o algoritmo não-recursivo volta a se destacar e
obtém o menor número de comparações com uma boa margem de diferença. Em
segundo lugar, o clássico, o M3 e o QPE mostram números similares entre
si, e os mistos de inserção apresentam o pior desempenho em termos de
comparações, com piora crescente na medida em que o percentual mínimo
para iniciar a inserção aumenta.

    \begin{center}
    \adjustimage{max size={0.9\linewidth}{0.9\paperheight}}{graficos-comparativos_files/graficos-comparativos_26_0.png}
    \end{center}
    { \hspace*{\fill} \\}
    
    O QPE volta a mostrar o pior desempenho em se tratando de comparações,
no caso de vetor ordenado de forma crescente; os mistos de inserção
apresentam a melhor eficiência, conseguindo um número até 3x menor que
os restantes; e o QC, QNR e QM3 se mostram entre os dois grupos, mas
consideravelmente mais próximos do caso de inserção.

    \begin{center}
    \adjustimage{max size={0.9\linewidth}{0.9\paperheight}}{graficos-comparativos_files/graficos-comparativos_28_0.png}
    \end{center}
    { \hspace*{\fill} \\}
    
    O aumento na quantidade mesclada de inserção volta a aprimorar o
quicksort no caso em que o vetor está ordenado de forma decrescente; o
grupo QC/QM3/QNR apresentam volumes de 2 a 3 vezes maiores que os
mistos/inserção; e o QPE mostra o maior volume de comparações.

    \section{Conclusões}\label{conclusuxf5es}

O quicksort não-recursivo possui o melhor caso médio, apresentando bom
desempenho em termos de comparações, movimentações e de tempo, tendo em
vista que ele utiliza quantidade menor da pilha na memória que os outros
algoritmos. Além disso, ele é estável e varia pouco nos casos extremos
(vetor ordenado crescente / decrescente), mostrando-se bastante robusto
em comparação com os outros algoritmos. O quicksort com pivô no primeiro
elemento é o menos robusto, que pode estourar consideravelmente a
performance em casos extremos, mas não apresenta nenhuma melhoria
comparado a outros algoritmos mais robustos quando a entrada é
aleatória.

Os casos mesclados com inserção são mais eficazes que o restante quando
a entrada está razoavelmente organizada (em ordem crescente ou
decrescente), mas são bastante lentos quando a entrada é aleatória. Em
contrapartida, o quicksort clássico, o de mediana 3 e o não-recursivo
apresentam maior robustez que depende menos da organização do vetor de
entrada. O QPE é o menos confiável, enquanto o QNR, o QC e o QM3
mostram-se os mais robustos e eficientes no caso médio. O uso mesclado
com inserção pode ser interessante em casos onde se sabe que o vetor de
entrada está consideravelmente (mas não inteiramente) ordenado; fora
isso, o grupo QNR/QC/QM3 são os mais indicados (em especial, o QNR, por
consumir menos tempo e menos memória no geral, apresentando menor chance
de estourar a pilha).

    \section{Bibliografia}\label{bibliografia}

Os algoritmos de ordenação e da estrutura de dados do tipo pilha foram
implementados com base no livro do Cornen e nos slides passados em aula,
mas adaptados para o trabalho em uma classe Quicksort que reaproveitava
o código 'particao' para diferentes pivôs e aceitava o pivô como
entrada, para evitar redundância de código.


    % Add a bibliography block to the postdoc
    
    
    
    \end{document}
